%!TEX root = ../template.tex
%%%%%%%%%%%%%%%%%%%%%%%%%%%%%%%%%%%%%%%%%%%%%%%%%%%%%%%%%%%%%%%%%%%%
%% abstract-en.tex
%% NOVA thesis document file
%%
%% Abstract in English
%%%%%%%%%%%%%%%%%%%%%%%%%%%%%%%%%%%%%%%%%%%%%%%%%%%%%%%%%%%%%%%%%%%%

\typeout{NT FILE abstract-en.tex}%

%%Abstract um resumo do tema, 1 pagina
%%Paragrafo 

Verification as we know is a very important area specially for critical systems that envolve high levels of trust. Tools
like \whythree allow specification and automated verification by delegating proof obligations to external proovers. On the other
hand, \cml offers a fully verified compilation chain based on Standard ML operation semantics, providing a path from source code 
with formal annotations to executable code with formal guarantees.

The main objective for this work is to explore, develope and expand a verification pipeline that starts from programs written
on \ocaml with \gospel annotations, passing through the automatic translation to \whyml through \cameleer, where verification 
is performed, and concludes on the generation of equivalent code in \cml. The goal is to ensure the extracted code preserves 
the verified properties, promoting a continuous formal verification throughout the process. Additionally, the pipeline will 
support translating the provided \cml code into \ocaml code.

The work to be done inclueds analizing and adapting the extractor already provided by \cameleer in order to allow a better 
conversion from \ocaml code into \cml code, respecting the syntactic and semantic differences between them. Mechanisms will be
implemented to detect and report cases where translation is not possible because of incompatabilities between models.

In the preparation fase, a thorough study was conducted about formal operacional semantics through Hoare Logic,
Weakest preconditions and certified compilers. Furthermore, the internal workings of tools such as \whythree, \cameleer 
and \cakeml were analyzed, while identifying which of the steps from the pipeline require changes or integrations.

