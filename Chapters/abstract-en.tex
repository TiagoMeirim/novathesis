%!TEX root = ../template.tex
%%%%%%%%%%%%%%%%%%%%%%%%%%%%%%%%%%%%%%%%%%%%%%%%%%%%%%%%%%%%%%%%%%%%
%% abstract-en.tex
%% NOVA thesis document file
%%
%% Abstract in English
%%%%%%%%%%%%%%%%%%%%%%%%%%%%%%%%%%%%%%%%%%%%%%%%%%%%%%%%%%%%%%%%%%%%

\typeout{NT FILE abstract-en.tex}%

The study of the Formal Verification field is very important for critical systems that
involve high levels of trust. A significant advancement in recent research has been the
development of certified compilers, such as \cml, which ensure that the generated 
machine code preserves the behaviour of the original program. The integration of these 
properties in a pipeline with previously verified code can provide powerful correctness
guarantees. 

The main objective of this work is to explore and develop a verification pipeline 
that starts with programs written in \ocaml with \gospel annotations with the end goal of
producing correct-by-construction \cml code. The first step involves translating the
annotated \ocaml code to \whyml using the \cameleer tool. This \whyml code should then be 
verified on the \whythree platform and ultimately extracted to \cml. Additionally, the 
pipeline should also feature a translation scheme in the opposite direction.

Currently, the extraction mechanism in \whythree to \cml is outdated and only supports
a subset of the language. As such, we intend to revisit and expand upon it in this 
work, by updating the translation scheme, implement a mechanism to stop the extraction
of non-verified code and to report cases where translation is not possible due to 
incompatibilities between features.

In this document, a thorough study was conducted concerning the state of \whythree's
extraction mechanism, alongside the selected tools, to identify which steps of the 
pipeline require modifications. Moreover, we also present the theoretical 
concepts concerning Operational Semantics, Deductive Verification and Verified 
Compilers.


\keywords{
  Deductive Verification \and
  Certified Compilers \and
  Verification Pipeline \and
  \cml \and
  \cameleer \and
  \gospel \and
  \ocaml \and
  \whythree
}

