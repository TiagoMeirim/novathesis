%!TEX root = ../template.tex
%%%%%%%%%%%%%%%%%%%%%%%%%%%%%%%%%%%%%%%%%%%%%%%%%%%%%%%%%%%%%%%%%%%
%% chapter1.tex
%% NOVA thesis document file
%%
%% Chapter with introduction
%%%%%%%%%%%%%%%%%%%%%%%%%%%%%%%%%%%%%%%%%%%%%%%%%%%%%%%%%%%%%%%%%%%

\typeout{NT FILE chapter1.tex}%

\chapter{Introduction}
\label{cha:introduction}

% This work is motivated by the importance of ensuring correctness in critical systems and the recent developments 
% in formal verification with certified compilers. Which, we present, in detail, below:

\section{Motivation}
\label{sec:Motivation}


Progress in Formal Verification has steadily been made over the years. 
The foundational ideas emerged as early as 1937 by Alan Turing~\cite{Turing37} and by 1949 a program
represented as a flowchart was formally verified~\cite{early-proof}. Other notable papers where mathematical propositions 
established purely through theoretical reasoning include works by Alan Turing~\cite{Turing38}, Church
~\cite{church-unsolvableproblemof-1936} and Gödel~\cite{ZACH2005917}. These results laid important 
groundwork in logic, computation, and formal reasoning without relying on physical implementation or empirical methods.
Significant progress was made in the 1960s with the introduction of formal systems to reason about program correctness. 
Floyd's work on flowcharts~\cite{Floyd93} and Hoare's in programs~\cite{Hoare69} marked a shift toward systematic verification of 
program behaviour using logical assertions.
In recent years, two major developments have transformed the landscape of formal verification. The first is the so-called SMT~\cite{abs-1711-10641} 
(Satisfiability Modulo Theories) revolution, which has enabled powerful automated verification tools capable of discharging 
logical obligations without human intervention~\cite{Barrett2018}. The second is the emergence of certified compilers, such as 
CompCert~\cite{Leroy09-back-end}, which provably ensures that the generated machine code faithfully preserves the semantics of the source 
program.

Why are verified compilers such an important target for formal verification? If a verified program is compiled by a 
faulty compiler, the resulting executable may not preserve its intended behaviour, invalidating any higher-level correctness
obtained previously. Compilers like CompCert and \cml~\cite{POPL14} address this issue by providing formally verified compilation pipelines
~\cite{LoowKTMNAF19, GrossEPPC22, Leroy09}, thereby eliminating a major source of uncertainty and ensuring that correctness is 
preserved from source code to machine code~\cite{LeroyCompilerMeaning}.

Developments in relation to verified code have become increasingly crucial to achieve correctness and provide safety, 
the way we define correctness has more than one definition, it can be specified informally or written in formal language 
~\cite{Filliatre11}. The weight of functional languages for deductive software verification has been quite low despite having
good candidates for verification, like \ocaml. Ever since 2018 with the introduction of \gospel~\cite{ChargueraudFLP19}, a
formal specification language designed specifically for \ocaml, verification has become easier with modular specification
that doesn't need many changes to \ocaml code. This wasn't the first time formal logic and proof has been merged with 
functional programming, previous and more foundational systems like Coq~\cite{75277.75285} and \whyml~\cite{FilliatreP13} have
paved the way. With the development of a behavioural specification language for \ocaml, soon after a deductive verification tool
was created for this language, that being \cameleer~\cite{PereiraR20}, a tool that translates a formally-specified program into 
corresponding code in \whyml.

With the introduction of certified compilers and state-of-the-art automated verification tools, it is possible to merge them in
a pipeline to create correct high-level programs that when compiled that are not faulty.
This is a highly trustworthy deductive verification system, that benefits from the features and correctness guarantees from each 
component of the pipeline throughout all the steps. Our motivation stems from the insurance that such a pipeline could provide 
to the verification of \ocaml programs with \gospel annotations that eventually compile to correct machine code, by using the \cml
compiler.

It is also important to distinguish two broad methodologies in deductive verification. On one hand, extraction-based verification involves 
developing verified programs directly within a proof assistant and then extracting executable code, ensuring correctness by construction. 
On the other hand, translation-based verification starts from existing codebases, often annotated with formal specifications, and translates 
them into an intermediate verification language. Verification conditions are then generated and discharged by automated or interactive 
theorem provers~\cite{Leroy09}.

\section{Problem Definition}
\label{sec:Problem_Definition}

In order to create a pipeline from \ocaml to \cml and back, there is the need to understand the already existing translation tools.
Firstly, \cameleer takes as input \ocaml programs with \gospel annotations and internally these are translated onto \whyml code, which
can then be verified inside \whythree's ide or extracted to another language. The \whythree extraction mechanism currently allows the 
translation from \whyml to four other languages, with one of them being \cml. By combining these two mechanisms we reach our goal
of creating a pipeline to produce compiled code completely verified. However, we observed that the current extraction mechanism 
concerning \cml is severely outdated and unreliable, after extensive tests using multiple \ocaml features since it failed to compile 
on several instances. \cameleer also features an extraction mechanism that currently produces \ocaml code from its intermediary
representation, \whyml, instead of producing \cml code. Another facet of this pipeline is to translate \cml programs into \ocaml
code, and surprisingly, there is no such tool. The development of this new tool should enable direct verification of \cml code, 
allowing correctness to be established before any extraction process takes place. This leads us to the following question: 

\vspace{5mm}
\centerline{\textit{How do extraction-based and translation-based approaches differ in ensuring program correctness?}}

In our proposed pipeline, both extraction-based and translation-based approaches are integrated, offering a unique opportunity 
to examine how these methodologies apply for the same case studies. This dual exploration not only highlights the distinctions 
between the two, but also leads to reflection on:

\begin{enumerate}
    \item \textit{How do we bridge these tools to create a pipeline that ensures correctness throughout all steps?}
    \item \textit{How does adding a specification language affect the verification workflow of certified compilers?}
\end{enumerate}

\section{Goals and Expected Contributions}
\label{sec:Goals_and_Expected_Contributions}

The first goal in this work is to have a detailed look into the features of both \ocaml and \cml languages, find the notable 
differences and absences, to know which features to test using the extraction mechanism. The next goal should be to change 
internally the source code of the \cameleer tool to support the extraction of \cml code. With this, we should be able to identify
in detail the issues with the translation, what are the necessary changes to compile correctly, which may include introducing new features
or fixing syntactic issues for the features already supported by the extraction mechanism, directly in \whythree source code.
For the features that are not supported by one of the languages, providing a comprehensive error message rather than allowing 
the extraction to take place. To make the pipeline more robust we should confirm if the program is verified beforehand when performing 
an extraction. Since, there is no translation mechanism from \cml to \ocaml, it creates the need to develop a new tool. Our objective is 
to extend the \gospel parser to support languages within the ML family, enabling the application of formal specifications directly to 
\cml code. This will allow verification to take place prior to translation. Specifically, we aim to build a new translation tool, 
driven by a custom parser, that converts \cml with \gospel annotations into \ocaml with \gospel annotation. To achieve this, we define
a subset of the \cml code that respects all the syntactic and semantic differences, meanwhile flagging with an error message any 
unsupported features.

\section{Report Structure}
\label{sec:Report_Structure}

The document is divided by 5 chapters. 

In chapter \ref{cha:Background} it is possible to find the foundational theoretical concepts for Hoare Logic and an introduction to 
the languages and tools used for verification and extraction mechanism.

Chapter \ref{cha:State_Of_The_Art} describes the relevant pieces of work already available on the context of certified compilers, 
while also mentioning the current verification pipeline available using \cameleer, \whythree and \cml. 

Chapter \ref{cha:Preliminary_Results} displays thoroughly what the current tools do for various examples while also pointing out the 
places where they need improvement. 

Finally, chapter \ref{cha:Work_Plan} divides, in detail, what are the formal steps for our work and how they will be scheduled in 
the calendar.