%!TEX root = ../template.tex
%%%%%%%%%%%%%%%%%%%%%%%%%%%%%%%%%%%%%%%%%%%%%%%%%%%%%%%%%%%%%%%%%%%%
%% chapter5.tex
%% NOVA thesis document file
%%
%% Chapter with lots of dummy text
%%%%%%%%%%%%%%%%%%%%%%%%%%%%%%%%%%%%%%%%%%%%%%%%%%%%%%%%%%%%%%%%%%%%

\typeout{NT FILE chapter5.tex}%

\chapter{Work Plan}
\label{cha:Work_Plan}

\section{From OCaml to CakeML}

\subsection{Correct Boolean Literals Syntax}

In \cml, the boolean literals start with uppercase contrary to \ocaml. In this task we ought to revise the \whythree's extraction scheme,
which currently define these literals with all lowercase letters. So, in order to match boolean literals with \cml syntax, we must
capitalize the initial letters (e.g., \inlinecode{true} $\rightarrow$ \inlinecode{True}).

\subsection{Correct Reference Declaration Syntax}

Just as the boolean literals, references also start with uppercase which is also not the same as \ocaml. The same capitalization solution 
will be applied for reference declarations in order to follow \cml conventions (e.g., \inlinecode{ref} $\rightarrow$ \inlinecode{Ref}).

\subsection{Correct Let-Binding Syntax}

Regarding the let-binding structure in \cml, this alwasy terminates with the \inlinecode{end} keyword. During translation sometimes there 
are some missing \inlinecode{end} tokens. This may be due to \ocaml not having an explicit termination token for let-bindings. Additionally, 
the parameters of the functions are duplicated with let-binding, this is redundant, and it is our objective to simplify the resulting 
translations.

\subsection{Correct Data Structures Operations Syntax}

Two of the most commonly used libraries for data structures, List and Array, have displayed errors when translating their operations into 
\cml. Some of the errors include the omitted prefix for the library name and different operation names. 
Even if the operations are quite similar from \ocaml and \cml, the translation doesn't generate the same operation calling, therefore, to 
have a fully correct and compilable code in \cml arises the necessity to replace the data structures operations with the corresponding 
ones.

\subsection{Correct Data Type Syntax}

When defining the data types, the syntax from \ocaml and \cml have some differences when it comes to tuple declarations, in particular, 
the \inlinecode{of} keyword and the separator \inlinecode{*} found in \ocaml are not used in \cml. However, the translation mechanism 
misplaces the generic type before the \inlinecode{datatype} keyword in which the data type is written. The solution comes through fixing 
data type definitions to respect \cml ordering and syntax.

\subsection{Correct Exception Handling}

In \ocaml we can declare exceptions within functions, which is not possible in \cml. All exceptions need to be declared at the global 
level. Furthermore, \ocaml supports a few predefined exceptions found in the standard library which have no direct translation in
\cml, so they also need to be declared. For this task it is essential to refactor the exception declarations and usage to comply with 
\cml, ensuring all exceptions are defined properly.

\section{Translation from CakeML to OCaml}

\subsection{Arithmetic and Boolean operations}

The first step in any basic translation tool is to define the integer and boolean data types as well as their respective operations.
This includes defining the literals correctly, as previously mentioned the boolean tokens are not equally written, and operations
such as addition, subtraction, multiplication, equality and inequality. Whenever necessary adjusting casing and syntax will be performed.

\subsection{Let-Bindings}

The next for the translation tool is to support let-bindings. The correct syntax for this construct in \cml includes explicit 
termination with the \inlinecode{end} keyword, meanwhile in \ocaml there is no explicit termination. This construct requires 
special attention when multiple lets are nested due to the environments of the bindings.

\subsection{References}

To support references it is important to define three operations, these being creation, assignment and access. The correct translation
from \cml to \ocaml must take into consideration the main syntax difference which is casing: \inlinecode{ref} in \ocaml
and \inlinecode{Ref} in \cml.

\subsection{Data Structure Operations}

Most real programs make use of built-in complex data structures such as Arrays and Lists. This urges for the translation of their 
basic operations already present in the standard library, besides the additional operations featured in the respective dedicated 
library for that data structure.

\subsection{Data Types}

For many programs the built-in data structures does not suffice, so user defined data types are a must for these programming languages.
Such it is natural to provide translation scheme for the definition of data types in \cml into idiomatic \ocaml type declarations, 
preserving constructors and structure.

\subsection{Correct Exception Handling}

Robust programs are built with the expectation to fail occasionally, so there should be mechanisms to handle and recover from these 
situations. One of these mechanisms are exceptions, which offer a clean and concise flagging for these non-natural cases.
When transforming exception declarations and usage from \cml back into \ocaml style with inline definitions and raises, we must also take
into consideration that \cml only allows global declaration of exceptions.

\section{Dissertation}

\subsection{Writing}

In the last month, we will finalize the dissertation document by detailing the modifications made to the codebase, describing the 
implementation steps, and analysing the results and limitations of the pipeline and the translation tool.

\begin{figure}[h]
\tikzset{every picture/.style={xscale=0.65,yscale=0.65,transform shape}}
\begin{ganttchart}[ y unit chart = 0.6cm,
                    vgrid,
                    bar top shift=-0.1,
                    bar height=0.6,
                    title height=0.7]{1}{28}
    \gantttitle{August}{4}
    \gantttitle{September}{4}
    \gantttitle{October}{4}
    \gantttitle{November}{4}
    \gantttitle{December}{4}
    \gantttitle{January}{4}
    \gantttitle{February}{4}\\
    \gantttitlelist{1,...,4}{1}
    \gantttitlelist{1,...,4}{1}
    \gantttitlelist{1,...,4}{1}
    \gantttitlelist{1,...,4}{1}
    \gantttitlelist{1,...,4}{1}
    \gantttitlelist{1,...,4}{1}
    \gantttitlelist{1,...,4}{1} \\

\ganttgroup{From OCaml to CakeML}{1}{12}\\

\ganttbar{Correct Boolean Literals Syntax}{1}{1} \\

\ganttbar{Correct Reference Declaration Syntax}{2}{2} \\

\ganttbar{Correct Let-Binding Syntax}{3}{4} \\

\ganttbar{Correct Data Structures Operations Syntax}{5}{6} \\

\ganttbar{Correct Data Type Syntax}{7}{9} \\

\ganttbar{Correct Exception Handling}{10}{12} \\

\ganttgroup{Translation from CakeML to OCaml}{13}{26} \\

\ganttbar{Arithmetic and Boolean operations}{13}{14} \\

\ganttbar{Let-Bindings}{15}{16} \\

\ganttbar{References}{17}{19} \\

\ganttbar{Data Structure Operations}{20}{21} \\

\ganttbar{Data Types}{22}{23} \\

\ganttbar{Correct Exception Handling}{24}{26} \\

\ganttgroup{Dissertation}{25}{28} \\
\ganttbar{Writing}{25}{28}

\end{ganttchart}

\caption{Tentative Schedule}

\end{figure}
