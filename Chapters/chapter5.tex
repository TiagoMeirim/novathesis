%!TEX root = ../template.tex
%%%%%%%%%%%%%%%%%%%%%%%%%%%%%%%%%%%%%%%%%%%%%%%%%%%%%%%%%%%%%%%%%%%%
%% chapter5.tex
%% NOVA thesis document file
%%
%% Chapter with lots of dummy text
%%%%%%%%%%%%%%%%%%%%%%%%%%%%%%%%%%%%%%%%%%%%%%%%%%%%%%%%%%%%%%%%%%%%

\typeout{NT FILE chapter5.tex}%

\chapter{Work Plan}
\label{cha:Work_Plan}

\section{From OCaml to CakeML}

\subsection{Correct Boolean Literals Syntax}

Modifying boolean literals to match \cml syntax, capitalizing initial letters (e.g., \inlinecode{true} 
$\rightarrow$ \inlinecode{True}).

\subsection{Correct Reference Declaration Syntax}

Adjusting reference declarations to follow \cml conventions, including capitalization and syntax structure.

\subsection{Correct Let-Binding Syntax}

Ensuring all \inlinecode{let} bindings are properly closed with \inlinecode{end} as required by \cml.

\subsection{Correct Array Operations Syntax}

Replacing \ocaml array operations with their \cml equivalents.

\subsection{Correct Datatype Syntax}

Fixing datatype definitions to respect \cml ordering and syntax, especially for constructors with multiple arguments.

\subsection{Correct Exception Handling}

Refactoring exception declarations and usage to comply with \cml, ensuring all exceptions are defined at the top level.

\section{Translation from CakeML to OCaml}

\subsection{Arithmetic and Boolean operations}

Mapping \cml arithmetic and boolean operators to their \ocaml equivalents, adjusting casing and syntax as needed.

\subsection{Let-Bindings}

Converting \cml \inlinecode{let}... \inlinecode{end} structures into standard \ocaml let-bindings without explicit 
termination.

\subsection{References}

Translating \cml reference operations into \ocaml.

\subsection{Array Operations}

Adapting \cml array access and manipulation to \ocaml syntax.

\subsection{Datatypes}

Reconstructing datatype definitions from \cml into idiomatic \ocaml type declarations, preserving constructors 
and structure.

\subsection{Correct Exception Handling}

Transforming exception declarations and usage from \cml back into \ocaml style with inline definitions and raises.

\section{Dissertation}

\subsection{Writing}

Finalizing the dissertation document by detailing the modifications made to the codebase, describing the implementation 
steps, and analysing the results and limitations of the translation process.

\begin{figure}[h]
\tikzset{every picture/.style={xscale=0.65,yscale=0.65,transform shape}}
\begin{ganttchart}[ y unit chart = 0.6cm,
                    vgrid,
                    bar top shift=-0.1,
                    bar height=0.6,
                    title height=0.7]{1}{28}
    \gantttitle{August}{4}
    \gantttitle{September}{4}
    \gantttitle{October}{4}
    \gantttitle{November}{4}
    \gantttitle{December}{4}
    \gantttitle{January}{4}
    \gantttitle{February}{4}\\
    \gantttitlelist{1,...,4}{1}
    \gantttitlelist{1,...,4}{1}
    \gantttitlelist{1,...,4}{1}
    \gantttitlelist{1,...,4}{1}
    \gantttitlelist{1,...,4}{1}
    \gantttitlelist{1,...,4}{1}
    \gantttitlelist{1,...,4}{1} \\

\ganttgroup{From OCaml to CakeML}{1}{12}\\

\ganttbar{Correct Boolean Literals Syntax}{1}{1} \\

\ganttbar{Correct Reference Declaration Syntax}{2}{2} \\

\ganttbar{Correct Let-Binding Syntax}{3}{4} \\

\ganttbar{Correct Array Operations Syntax}{5}{6} \\

\ganttbar{Correct Datatype Syntax}{7}{9} \\

\ganttbar{Correct Exception Handling}{10}{12} \\

\ganttgroup{Translation from CakeML to OCaml}{13}{26} \\
\ganttbar{Arithmetic and Boolean operations}{13}{14} \\

\ganttbar{Let-Bindings}{15}{16} \\

\ganttbar{References}{17}{19} \\

\ganttbar{Array Operations}{20}{21} \\

\ganttbar{Datatypes}{22}{23} \\

\ganttbar{Correct Exception Handling}{24}{26} \\

\ganttgroup{Dissertation}{25}{28} \\
\ganttbar{Writing}{25}{28}

\end{ganttchart}

\caption{Tentative Schedule}

\end{figure}
