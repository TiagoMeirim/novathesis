%!TEX root = ../template.tex
%%%%%%%%%%%%%%%%%%%%%%%%%%%%%%%%%%%%%%%%%%%%%%%%%%%%%%%%%%%%%%%%%%%%
%% chapter5.tex
%% NOVA thesis document file
%%
%% Chapter with lots of dummy text
%%%%%%%%%%%%%%%%%%%%%%%%%%%%%%%%%%%%%%%%%%%%%%%%%%%%%%%%%%%%%%%%%%%%

\typeout{NT FILE chapter5.tex}%

\chapter{Work Plan}
\label{cha:Work_Plan}

In this chapter we enumerate all the steps for the next few months about the development and objectives to be done for our work.
Every task is put onto a flexible time frame, because some tasks may end sooner or later than we expect, leading us to adjust the 
schedule when needed.

\section{From OCaml to CakeML}

The first part for this work is focusing on improving the extraction mechanism, while fixing the issues aforementioned.

\begin{itemize}

\item \textbf{Correct Extraction Syntax -} In this task we ought to revise the \whythree's extraction scheme, as mentioned in the case
studies the main changes have been identified in order to have an updated extraction mechanism from \ocaml with \gospel into \cml. Most
changes will be focused on the syntax generated from the extraction, so the syntaxes to be updated will be the boolean literals, reference
declarations, let-bindings, data structures operations, data types and exception handling. What will be done in this section is changing
case letters, add and eliminate tokens, change scoping, change declarations and operation calling. These changes need to be done because 
the syntax of \ocaml is quite different from \cml, and since the extraction mechanism is outdated all of the above are essential.


\item \textbf{Define An Extensive Case Study -} Develop a library to 

\item \textbf{Extensively Test Extraction Mechanism -} Create an extensive battery test to ensure all the changes done to the extraction
mechanism are correct. Use the tests from the case studies and analyse the code generated from \ocaml with \gospel annotations and
confirm if it indeed maintains the same behaviour and if the code is compilable for \cml.


% which currently define these literals with all lowercase letters. So, in order to match boolean literals with \cml syntax, we must
% capitalize the initial letters (e.g., \inlinecode{true} $\rightarrow$ \inlinecode{True}).

% \item \textbf{Correct Reference Declaration Syntax -} Similar to boolean literals, reference declarations also start with an uppercase 
% letter, which contrasts with \ocaml. The same 
% capitalization solution will be applied to the reference token in order to follow \cml conventions 
% (e.g., \inlinecode{ref} $\rightarrow$ \inlinecode{Ref}).

% \item \textbf{Correct Let-Binding Syntax -} Regarding the let-binding structure in \cml, this always terminates with the \inlinecode{end} 
% keyword. During translation sometimes there 
% are some missing \inlinecode{end} tokens. This may be due to \ocaml not having an explicit termination token for let-bindings. Additionally, 
% the parameters of the functions are duplicated with let-bindings, this is redundant, and it is our objective to simplify the resulting 
% translations.

% \item \textbf{Correct Data Structures Operations Syntax -} Two of the most commonly used libraries for data structures, List and Array, 
% have displayed errors when translating their operations into 
% \cml. Some of the errors include the omitted prefix for the library name and different operation names. 
% Even if the operations are quite similar from \ocaml and \cml, the translation doesn't generate the same operation calling, therefore, to 
% have a fully correct and compilable code in \cml arises the necessity to replace the data structures operations with the corresponding 
% ones.

% \item \textbf{Correct Data Type Syntax -} When defining the data types, \cml's syntax differs from \ocaml's when it comes to 
% tuple declarations, in particular, 
% the \inlinecode{of} keyword and the separator \inlinecode{*} found in \ocaml are not used in \cml. Another translation error consists in
% misplacement of the generic type, which is situated before the \inlinecode{datatype} keyword, rather than after. The solution comes 
% through fixing data type definitions to respect \cml ordering and syntax.

% \item \textbf{Correct Exception Handling -} In \ocaml we can declare exceptions within functions through let-bindings, contrary to
% \cml. All exceptions need to be 
% declared at the global level. Furthermore, \ocaml supports a few predefined exceptions found in the standard library which have no 
% direct translation in \cml, so they also need to be declared. For this task it is essential to refactor the exception declarations 
% and usage to comply with \cml, ensuring all exceptions are defined properly.

\end{itemize}

\section{Translation from CakeML to OCaml}

The next big task is to create a new tool that can do the inverse pipeline of the previous work. This new tool will respect
the syntactic and semantic differences while translating, and the available features of \cml.

\begin{itemize}

\item \textbf{Functional Constructs -}

\item \textbf{Imperative Constructs -}

\item \textbf{Module System -}

\item \textbf{Extensively Test The New Tool -}


% \item \textbf{Arithmetic and Boolean operations -} The first step in any basic translation tool is to define the integer and boolean 
% data types as well as their respective operations.
% This includes defining the literals correctly, as previously mentioned the boolean tokens are not equally written, and operations
% such as addition, subtraction, multiplication, equality and inequality. Whenever necessary, casing and syntax will be adjusted.

% \item \textbf{Let-Bindings -} This construct is essential to functional languages. The correct syntax for this construct in 
% \cml includes explicit 
% termination with the \inlinecode{end} keyword, meanwhile in \ocaml there is no explicit termination. This construct requires 
% special attention when multiple let-bindings are nested due to the different environments.

% \item \textbf{References -} To support references it is important to define three operations, these being creation, assignment and 
% access. The correct translation
% from \cml to \ocaml must take into consideration the main syntax difference which is casing: \inlinecode{ref} in \ocaml
% and \inlinecode{Ref} in \cml.

% \item \textbf{Data Structure Operations -} Most real programs make use of built-in complex data structures such as Arrays and Lists. 
% This urges for the translation of their 
% basic operations already present in the standard library, besides the additional operations featured in the respective dedicated 
% library for that data structure.

% \item \textbf{Data Types -} For many programs the built-in data structures do not suffice, so, user defined data types are a must 
% for programming languages.
% Such that, it is natural to provide a translation scheme for the definition of data types in \cml into idiomatic \ocaml type declarations, 
% preserving constructors and structure.

% \item \textbf{Exception Handling -} Robust programs are built with the expectation to fail occasionally, so there should 
% be mechanisms to handle and recover from these 
% situations. One of these mechanisms are exceptions, which offer clean and concise flagging for these non-natural cases.
% When transforming exception declarations and usage from \cml back into \ocaml style with inline definitions and raises, we must also take
% into consideration that \cml only allows global declaration of exceptions.

\end{itemize}

\section{Dissertation}

\begin{itemize}

\item \textbf{Writing -}
In the last month, we will finalize the dissertation document by detailing the modifications made to the codebase, describing the 
implementation steps, and analysing the results and limitations of the pipeline and the translation tool.

\end{itemize}

%% \newpage

\section{Gantt Chart}

Below we present a Gantt Chart with the previously mentioned tasks spread into the next seven months.

\begin{figure}[ht]
\tikzset{every picture/.style={xscale=0.65,yscale=0.65,transform shape}}
\begin{ganttchart}[ y unit chart = 0.6cm,
                    vgrid,
                    bar top shift=-0.1,
                    bar height=0.6,
                    title height=0.7]{1}{28}
    \gantttitle{August}{4}
    \gantttitle{September}{4}
    \gantttitle{October}{4}
    \gantttitle{November}{4}
    \gantttitle{December}{4}
    \gantttitle{January}{4}
    \gantttitle{February}{4}\\
    \gantttitlelist{1,...,4}{1}
    \gantttitlelist{1,...,4}{1}
    \gantttitlelist{1,...,4}{1}
    \gantttitlelist{1,...,4}{1}
    \gantttitlelist{1,...,4}{1}
    \gantttitlelist{1,...,4}{1}
    \gantttitlelist{1,...,4}{1} \\

\ganttgroup{From OCaml to CakeML}{1}{12}\\

\ganttbar{Correct Extraction Syntax}{1}{3} \\

\ganttbar{Define An Extensive Case Study}{4}{10} \\

\ganttbar{Extensively Test Extraction Mechanism}{11}{12} \\

\ganttgroup{Translation from CakeML to OCaml}{13}{26} \\

\ganttbar{Functional Constructs}{13}{16} \\

\ganttbar{Imperative Constructs}{17}{21} \\

\ganttbar{Module System}{22}{24} \\

\ganttbar{Extensively Test The New Tool}{25}{26} \\

\ganttgroup{Dissertation}{25}{28} \\
\ganttbar{Writing}{25}{28}

\end{ganttchart}

\caption{Tentative Schedule}

\end{figure}
