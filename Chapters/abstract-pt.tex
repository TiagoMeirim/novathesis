%!TEX root = ../template.tex
%%%%%%%%%%%%%%%%%%%%%%%%%%%%%%%%%%%%%%%%%%%%%%%%%%%%%%%%%%%%%%%%%%%%
%% abstract-pt.tex
%% NOVA thesis document file
%%
%% Abstract in Portuguese
%%%%%%%%%%%%%%%%%%%%%%%%%%%%%%%%%%%%%%%%%%%%%%%%%%%%%%%%%%%%%%%%%%%%

\typeout{NT FILE abstract-pt.tex}%

A verificação formal de programas é uma área muito importante, especialmente em sistemas críticos que exigem 
alto grau de confiança. Ferramentas como \whythree permitem a especificação e verificação automática de programas 
através da geração de condições de verificação, que são posteriormente delegadas a provadores externos. Por outro lado, 
\cml oferece uma cadeia de compilação completamente verificada, baseada numa semântica operacional formal de Standard ML,
permitindo uma verificação de ponta a ponta, desde o código fonte com anotações formais até código executável com garantias 
formais.

O principal objetivo desta dissertação é explorar uma pipeline de verificação que parte de programas escritos em OCaml, 
anotados com especificações em \gospel, passando pela tradução automática para \whyml, onde a verificação é efetuada, culminando 
na geração de código equivalente em \cml. Pretende-se garantir que o código extraído preserva as propriedades verificadas, 
promovendo uma verificação formal contínua ao longo de todo o processo. Adicionalmente, a pipeline vai suportar tradução 
de códigoem  \cml para código em \ocaml.

O trabalho a realizar envolve a análise e adaptação da ferramneta de extração já existente no Cameleer, de forma a permitir 
a conversão segura de \ocaml para \cml, respeitando as diferenças sintáticas e semânticas entre as linguagens. Serão também 
implementados mecanismos de deteção e sinalização de casos em que a tradução não é possível devido a incompatibilidades entre 
os modelos de execução.

Na fase de preparação, foi realizado um estudo aprofundado sobre semânticas operacionais formais, na lógica de Hoare, 
na utilização de pré-condições fracas e compiladores verificados. Suplementarmente, o estudo do funcionamento interno das 
ferramentas \whythree, \cameleer e \cml serviu para identificar quais as etapas da pipeline que requerem alterações ou integração.


\keywords{
  Primeira palavra-chave \and
  Outra palavra-chave \and
  Mais uma palavra-chave \and
  A última palavra-chave
}