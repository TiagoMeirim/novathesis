%!TEX root = ../template.tex
%%%%%%%%%%%%%%%%%%%%%%%%%%%%%%%%%%%%%%%%%%%%%%%%%%%%%%%%%%%%%%%%%%%%
%% abstract-pt.tex
%% NOVA thesis document file
%%
%% Abstract in Portuguese
%%%%%%%%%%%%%%%%%%%%%%%%%%%%%%%%%%%%%%%%%%%%%%%%%%%%%%%%%%%%%%%%%%%%

\typeout{NT FILE abstract-pt.tex}%

O estudo da Verificação Formal é um campo muito importante para sistemas críticos que
envolvem confiança de alto nível. Um avanço significativo na investigação recente tem sido o
desenvolvimento de compiladores certificados, como o \cml, que assegura que o código máquina
gerado vai preservar o comportamento do programa original. A integração destas propriedades
numa \textit{pipeline} com código previamente verificado pode fornecer fortes garantias de correção.

O principal objetivo deste trabalho é explorar e desenvolver uma \textit{pipeline} de Verificação
onde começa com programas escritos em \ocaml com anotações em \gospel, onde o objetivo final
é produzir código \cml correto por construção. O primeiro passo envolve a tradução de código 
\ocaml anotado para \whyml através da ferramenta \cameleer. Este código \whyml deve ser 
posteriormente verificado na plataforma \whythree e em última instância extraído para \cml.
Adicionalmente, a \textit{pipeline} deve também conter um esquema de tradução na direção oposta,
onde propomos a utilização de \gospel como uma linguagem genérica de especificação para linguagens
da família ML, em particular \cml, de modo a verificar programas nestas linguagens utilizando
a ferramenta \cameleer.

Atualmente, o mecanismo de extração do \whythree para \cml é obsoleto e apenas suporta
um subconjunto da linguagem. Como tal, tencionamos revisitar e expandí-lo neste trabalho,
ao atualizar o esquema de tradução, implementar o mecanismo para parar a extração de
código não verificado e reportar casos onde a tradução não é possível devido a incompatibilidades
entre funcionalidades. Na direção oposta pretendemos desenvolver um esquema de tradução de 
\cml para \ocaml e reutilizar o parser de \gospel de modo a permitir verificação através de \cameleer.

Neste documento, foi conduzido um estudo minucioso sobre o estado do mecanismo de
extração do \whythree, em conjunto com as ferramentas escolhidas, para identificar quais 
os passos da \textit{pipeline} requerem modificações. Além disso, apresentamos
os conceitos teóricos sobre as Semânticas Operacionais, Verificação Dedutiva e Compiladores
Verificados.


\keywords{
  Verificação Dedutiva \and
  Compiladores Certificados \and
  \textit{Pipeline} de Verificação\and
  \cml \and
  \cameleer \and
  \gospel \and
  \ocaml \and
  \whythree
}