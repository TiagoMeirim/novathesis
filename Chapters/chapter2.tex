%!TEX root = ../template.tex
%%%%%%%%%%%%%%%%%%%%%%%%%%%%%%%%%%%%%%%%%%%%%%%%%%%%%%%%%%%%%%%%%%%%
%% chapter2.tex
%% NOVA thesis document file
%%
%% Chapter with the template manual
%%%%%%%%%%%%%%%%%%%%%%%%%%%%%%%%%%%%%%%%%%%%%%%%%%%%%%%%%%%%%%%%%%%%

\typeout{NT FILE chapter2.tex}%


\chapter{Background}
\label{cha:Background}

% Syntax & semantics
% Interpretation of PLs
% Compilation of PLs
% Type Systems
% Parsing
% Compiler otimizations (?)
% Hoare logic
% OCaml
% Standard ML
% Why3
% Cameleer

\section{Hoare logic}
\label{sec:Hoare_logic}

Hoare Logic is a formal system for reasoning about the correctness of computer programs. However, 
computer arithmetic often differs from the standard arithmetic familiar to mathematicians due to issues like finite 
precision, overflows, and machine-specific behaviors. To account for these differences, C.A.R. Hoare introduced a new logic
based on assertions and inference rules for reasoning about the partial correctness of programs. Drawing inspiration 
from mathematical axioms and formal proof techniques, he proposed a framework where program behavior could be specified 
and verified using logical formulas. This laid the foundation for systematic program verification and emphasized the need 
to model computational constraints, such as those arising from the limitations of machine arithmetic, within a formal system.

Hoare Logic, introduced by C.A.R. Hoare in his seminal 1969 paper, is a formal system designed to reason about the partial 
correctness of computer programs. Inspired by mathematical axioms and formal proof methods, it uses assertions and inference 
rules to specify and verify program behavior. Hoare's framework laid the foundation for systematic program verification and 
highlighted the importance of modeling low-level computational constraints, such as finite precision, overflows, and 
machine-specific arithmetic within formal systems.

"The purpose of this study is to provide a logical basis for proofs of the properties of a program"

\textbf{Assignment Axiom}

\[ \{P_0\} x := f \{P\} \]

where
\begin{itemize}
  \item{$x$ is a variable identifier;}
  \item{$f$ is an expression;}
  \item{$P_0$ is obtained from $P$ by substituting $f$ for all occurrences of $x$}
\end{itemize}

The axiom expresses that to prove a postcondition $P$ holds after assigning the expression $f$ to the variable $x$,
it suffices to prove the precondition $P_0$ before the assignment, where $P_0$ is obtained by substituting every occurrence of
$x$ in $P$ with the expression $f$.

\textbf{Rule of Composition}

\[ \vdash P\{Q_1\}R_1 \quad \text{and} \quad \vdash R_1\{Q_2\}R \quad \text{then} \quad \vdash P\{(Q_1 ; Q_2)\}R \]

The inference rule for composition states that if the postcondition of the first program segment matches the precondition 
of the second, then the entire program will produce the intended result—assuming the initial precondition of the first 
segment holds.



%%https://dl.acm.org/doi/pdf/10.1145/363235.363259
%%https://scispace.com/pdf/soundness-and-completeness-of-an-axiom-system-for-program-3g7csch545.pdf
%%https://cdn.bookey.app/files/pdf/book/en/a-discipline-of-programming.pdf

\section{OCaml}
\label{sec:OCaml}

%% Exemplos das features do OCaml
%% Falar de features concretas do OCaml
%% Exemplo place holder

\section{Standard ML}
\label{sec:Standard_ML}

%% Falar de semanticas operacionais

\section{Why3}
\label{sec:Why3}

\section{Cameleer}
\label{sec:Cameleer}

Cameleer ~\cite{PereiraR20}

\begin{gospell}
  let f x = x + 1 (*@ res = f x ensures res = x + 1)
\end{gospell}