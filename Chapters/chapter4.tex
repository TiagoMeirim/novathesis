%!TEX root = ../template.tex
%%%%%%%%%%%%%%%%%%%%%%%%%%%%%%%%%%%%%%%%%%%%%%%%%%%%%%%%%%%%%%%%%%%%
%% chapter4.tex
%% NOVA thesis document file
%%
%% Chapter with lots of dummy text
%%%%%%%%%%%%%%%%%%%%%%%%%%%%%%%%%%%%%%%%%%%%%%%%%%%%%%%%%%%%%%%%%%%%

\typeout{NT FILE chapter4.tex}%

\chapter{Preliminary Results}
\label{cha:Preliminary_Results}

%%mencionar que alterei a ferramenta do cameleer(adicionei a extracao direta de cameleer para cakeml)
%%Seccao para cada exemplo, apresentar codigo ocaml verificado, codigo traduzido e codigo corrigido se necessario
%%explicar o pedaco de codigo, condicoes do GOSPEL
%%explicar o que alteramos

%% Pequena introduçao sobre o problema, quais as features a espera de testar
%% Depois Apresentar codigo verificado em ocaml
%% Explica o codigo ainda por explicar e a especificaçao gospel
%% Mostrar traducao e explicar onde esta errado 
%% Mostrar o correto e por baixo explicar o que foi alterado para estar correto

\section{Tool Modifications}

As shown in Figure \ref{fig:Cameleer_pipeline}, which outlines the \cameleer pipeline, there is no explicit support for \cml. 
However, a small modification to the file \texttt{/bin/cli.ml} allows extending the existing extraction process. By changing the 
\inlinecode{execute\_extract} function to invoke \inlinecode{why3 extract -D cakeml {programa}.mlw -o {programa}.cml}, 
the tool can target \cml as a backend. With this adjustment, running \inlinecode{cameleer program.ml $--$extract} translates
the original \ocaml plus \gospel code into code in the \cml language. This enables extraction from \textsf{OCaml64} directly 
into \cml, providing the basis for a verified compilation pipeline.

The alternative to this process is to obtain the code translated \whyml from the flag \inlinecode{$--$debug} and then apply the 
mechanism of extraction of \whythree directly. Not only this process more complex, the debug flag produces code that doesn't 
let the launch of the provers from the \whythree ide because it contains proof goals directly in the code, which would imply 
also altering the code manually or to substantially modify the behaviour of the debug flag which is not intended.

%% Example to prove

%%Mencionar a alteracao do mecanismo de extracao do cameleer de ocaml64 para cakeml, internamente o cameleer usa o comando 
%%do why3 de extracao "why3 extract -D cakeml {programa}.mlw -o {programa}.cml"

\section{Case Studies}

For all the case studies below-mentioned, the \ocaml + \gospel code was compiled and verified through \whythree automatic
provers before applying the extraction command to translate into \cml. This verification step is essential because the correctness 
guarantees of the final \cml code depend entirely on the validity of the specifications and their successful discharge by the 
provers. Without ensuring that the \whyml intermediate code satisfies the given specifications, the generated \cml code would 
no longer carry the desired formal guarantees, thereby breaking the intended end-to-end verification pipeline.

Since this step is intended only to demonstrate what already exists, the generated code may contain significant syntax errors 
and may not be compilable. These issues reflect limitations in the current extraction process and highlight the need for 
further refinement.

\subsection{Tail Recursive Factorial}

Initially an iterative version of the factorial function was also considered, however, since \cml does not support imperative loop 
constructs such as \inlinecode{for} or \inlinecode{while} we had to look for alternative versions of this example. Instead, we opted 
to study the tail recursive variant to simulate iteration. This example highlights \ocaml's support for recursion through an auxiliary 
function, \inlinecode{fact\_aux}, which incrementally computes the factorial by accumulating the product of integers from 1 to n.

\begin{gospell}
let rec fact_aux n c t =
  if c <= n then fact_aux n (c+1) (t*c) else t
(*@
  r = fact_aux n c t
  requires n >= 0
  requires 0 < c <= n+1
  requires t = factorial (c-1)
  ensures r = factorial n
  variant n+1-c
*)
\end{gospell}

The auxiliary function \inlinecode{fact\_aux} receives as arguments the \inlinecode{n} value, that represents the factorial we want to 
compute, the \inlinecode{c} value, that represents the current index of the iteration, and \inlinecode{t} value, that is accumulating 
the result from previous iterations in order to display the solution.

The specifications have some pre-conditions, from those, the factorial to compute is higher or equal to 0, because the factorial as 
default is only calculated for non-negative values, the index of the iteration is between 0 and factorial to compute, because from the
code written when the index achieves the same value as the factorial to compute the function should terminate. For the post-conditions
we only need to make sure the result of the auxiliary function is indeed the factorial of n, which is calculated logically with the 
definition below. The variant serves to prove the termination of the recursive function with an expression that decreases every iteration
and is limited by 0. In this case since the counter is approaching n+1 every iteration, for the last iteration t holds the factorial of n,
which was the previous value of c and terminates successfully.

\begin{gospell}
(*@ 
  function rec factorial (n:int) :int = 
    if n=0 then 1 else n * factorial (n-1)
*)
(*@ 
  requires n >= 0
  variant n
*)
\end{gospell}

The definition of this function in \gospel represents the logical concept of a factorial, this can be used for later specifications 
to compare results. The implementation of this function is in a recursive way, since \gospel only supports the functional paradigm 
additionally logical definitions are meant to be simple rather than complex but optimized, since we want to simplify the proof.

\begin{gospell}
let fact n = fact_aux n 1 1
(*@ 
  r = fact n
  requires n >= 0 
  ensures r = factorial n
*)
\end{gospell}

The main function \inlinecode{fact} initializes the auxiliary call with appropriate base values, the first one is the counter
so it needs to start at 1 because for the multiplication with the accumulator starting at 0 would not take into consideration the 
first iteration. For the 1 in the accumulator, the third argument, represents the neutral element for multiplication. The \gospel 
annotations specify the expected behaviour, including preconditions such as non-negativity of n and the correctness of the returned 
result. 

This annotated version allows the function to be verified in \whythree before attempting any extraction. The extracted code can be
seen below:

\begin{cakeml}
fun fact_aux n c t = let val n1 = n in
  let val c1 = c in
  let val t1 = t in
  if c1 <= n1 then (fact_aux n1 (c1 + (1)) (t1 * c1))  else (t1)

fun fact n = let val n1 = n in fact_aux n1 (1) (1)
\end{cakeml}

The generated \cml code failed to compile due to differences in \inlinecode{let} expression syntax between \ocaml and \cml. In \cml, every 
\inlinecode{let} block must be explicitly closed with an \inlinecode{end} keyword, whereas in \ocaml, this is not required. It should
also be mentioned that the generated let-bindings generated are not necessary and only duplicate the variables in the arguments.

Initially we want to correct the let-binding termination as demonstrated in the following code:

\begin{cakeml}
fun fact_aux n c t = let val n1 = n in
  let val c1 = c in
  let val t1 = t in
  if c1 <= n1 then (fact_aux n1 (c1 + (1)) (t1 * c1))  else (t1)
  end end end

fun fact n = let val n1 = n in fact_aux n1 (1) (1) end
\end{cakeml}

However, our final goal is to eliminate as much unnecessary code as possible, in this case the expendable let-bindings, while 
maintaining the desired behaviour:

\begin{cakeml}
fun fact_aux n c t = 
  if c <= n then (fact_aux n (c + (1)) (t * c))  else (t)

fun fact n = fact_aux n (1) (1)
\end{cakeml}

Both versions of this example are compilable and function correctly. This first example is simple and quite clearly demonstrates that
our pipeline can function if small changes are applied to the extraction mechanism.

\subsection{Exception recursive search}

One possible use of exceptions, although it may seem natural at first, can be to stop iteration early in \ocaml for instance in the 
context of linear search when the desired element is found. Once again since for and while loops are not available in \cml to simulate 
iteration we use recursion with a counter:

\begin{gospell}
let rec search_aux a c n =
  let exception Break of int in try
    if c = Array.length a then raise Not_found else if a.(c) = n then raise (Break c) else search_aux a (c+1) n
  with Break i -> i
(*@
  r = search_aux a c n
  requires 0 <= c <= Array.length a
  requires forall k. 0 <= k < c -> a.(k) <> n
  raises Not_found -> forall k. 0 <= k < Array.length a -> a.(k) <> n
  ensures 0 <= r < Array.length a
  ensures a.(r) = n
  variant Array.length a - c
*)
\end{gospell}

The auxiliary function \inlinecode{search\_aux} receives as arguments an array, the current index and a value that is going to be 
compared on each iteration of the array's length. The program returns the index of the array in case it was found early by raising
the exception \inlinecode{Break}. By contrast, if the value is not in the array the exception \inlinecode{Not\_found} is raised at the end.

Inside the \gospel specifications, we must guarantee that index \inlinecode{c} is between values of 0 and the length of the array 
\inlinecode{a}, which in turn gives us the assurance that the array is never accessed for values out of its range expect for the case 
where \inlinecode{c} is equal to the length in which case the function terminates without accessing the array. To further strengthen the
precondition above there is the need to state that all the values previously seen can not be the desired value, otherwise the function
should have terminated earlier. Since the exception \inlinecode{Not\_Found} was not caught this means the function only returns a value
if this exception was not raised, therefore we only need to express two ensure clauses, one that states the resulting index is a value 
between 0 and the length of the array a minus one and that the element of the array in the index returned is effectively the value 
searching for. As the index is increasing in each iteration at some point it becomes equal to the length of the array thereby proving 
termination and the variant clause.

The core function, \inlinecode{linear\_search}, delegates the search process to the auxiliary function that recursively traverses 
the array:

\begin{gospell}
let linear_search a n = search_aux a 0 n
(*@
  r = linear_search a n
  raises Not_found -> forall k. 0 <= k < Array.length a -> a.(k) <> n
  ensures 0 <= r < Array.length a
  ensures a.(r) = n
*)
\end{gospell}

Following the same train of thought, we must express the two possible outcomes. The first is when the \inlinecode{Not\_found} exception
is raised and the element was not found within the boundaries of the array. The second case is when the result is in fact an integer
inside the boundaries and the value inside the array in the index provided is the same value being searched.

This makes the function verifiable by \whythree before attempting extraction to \cml, ensuring that key correctness properties are 
formally validated. The extracted code is:

\begin{cakeml}
fun search_aux a c n = let val a1 = a in
  let val c1 = c in
  let val n1 = n in
  let exception Break of (int) in
  ((if c1 = (a1.length) then (raise Not_found) 
    else (if (get a1 c1) = n1 then (raise (Break n1)) 
          else (search_aux a1 (c1 + (1)) n1)))
  handle   Break i => i)

fun linear_search a n =
  let val a1 = a in let val n1 = n in search_aux a1 (0) n1
\end{cakeml}

The generated \cml code fails to compile due to several incompatibilities between \ocaml and \cml that must be addressed 
to ensure a successful and semantically correct extraction. One of the primary issues concerns exception handling, as \cml 
requires all exceptions to be declared globally, outside the scope of functions, whereas the extracted code attempts to define 
the exception Break locally within a function using let-bound exception, a construct that is not valid in \cml. This syntactic 
difference results in a compilation error and reflects a broader incompatibility in exception scoping between the two languages.

Additionally, array operations in \cml differ substantially from those in \ocaml. While \ocaml allows for concise access through 
expressions like \inlinecode{a.(i)} while \cml demands the usage of a function from the array library \inlinecode{Array.sub a i}. 
The generated code fails in this aspect by using invalid expressions such as \inlinecode{a1.length} and \inlinecode{get a1 c1}, 
neither are supported by the \cml standard library nor array library. This divergence in the array library further contributes 
to the failure of the generated code to compile.

Moreover, the problems with let-bindings discussed earlier also contribute to another critical issue. These problems collectively 
illustrate the importance of a properly configured and semantically aware translation mechanism. The corrected code is presented below:

\begin{cakeml}

exception Not_found
exception Break int

fun search_aux a c n = let val a1 = a in
  let val c1 = c in
  let val n1 = n in
  ((if c1 = (Array.length a1) then (raise Not_found) 
    else (if (Array.sub  a1 c1) = n1 then (raise (Break n1)) 
          else (search_aux a1 (c1 + (1)) n1)))
  handle Break i => i)
  end end end

fun linear_search a n =
  let val a1 = a in let val n1 = n in search_aux a1 (0) n1 
  end end

\end{cakeml}

With this example we wanted to showcase how the extraction mechanism translates array definitions, access and other operations 
as well as handling exceptions.

\subsection{High-order}

In the documentation for \cml it was not possible to find any mention of higher-order functions as such we devised an example that
combines higher order with pattern matching and lists. The function in this example multiplies all the values of a list by a certain
integer:

\begin{gospell}
let rec mult_list l n =
    match l with
    | [] -> []
    | h :: t -> n * h :: mult_list t n
(*@ r = mult_list l n
        ensures r = map (fun x -> x * n) l
        variant l *)
\end{gospell}

This recursive function \inlinecode{mult\_list} receives as arguments a list and a value that will be used to multiply on each of the
lists values. While using a pattern matching the function will continue until the original list is empty.

The original list will have its values extracted one by one until the list is empty, signalling the function termination, which in turn
represents the variant clause. The function \inlinecode{map} is classical example of function programming that applies another function 
to all elements of a list. One can observe that the \inlinecode{mult\_list} function is a particular case of the \inlinecode{map} 
function, where the function is the multiplication by a given integer n and is expressed by an anonymous function. 

The logical implementation of the \inlinecode{map} function is presented below:

\begin{gospell}
(*@ function rec map (f: int -> int) (l: int list) : int list = 
        match l with
        | [] -> []
        | h::t -> (f h) :: map f t *)
(*@ variant l
        ensures List.length result = List.length l *)
\end{gospell}

The logical function \inlinecode{map} represents the application of a certain function to the values of the list provided. As the same
reason that variant clause is represented by the list, since the termination for this logical function is also determined by when the
list is empty. The is also the need to ensure that the length of the list resulting from applying the function to the original list
is the same since the number of values do not change with this transformation.

The corresponding generated code is:

\begin{cakeml}
fun mult_list l n = let val l1 = l in
    let val n1 = n in
    (case l1 of
        [] => []
    | h :: t => (n1 * h) :: (mult_list t n1))
\end{cakeml}

We can observe that pattern matching and basic list handling seems to be working well after the translation, however the extraction 
still generates clones of the arguments, the code can not be compilable. 

By solving the issues found, we tested the code and effectively, high-order functions are allowed by the compiler in \cml. The corrected 
code is displayed below:

\begin{cakeml}
fun mult_list l n = let val l1 = l in
    let val n1 = n in
    (case l1 of
        [] => []
    | h :: t => (n1 * h) :: (mult_list t n1))
    end end
\end{cakeml}

Since the issue continues to stem from the let-bindings, the solution to add the termination token or eliminating the unnecessary 
let-bindings still seems to solve the problem at hand.

\subsection{Depth-search tree}

User defined data types such as trees are very important for real world examples in functional programming, since traversing them
can easily be achieved recursively. 

\begin{gospell}
type 'a tree =
    | Leaf
    | Node of 'a tree * 'a * 'a tree
\end{gospell}

A tree can either be a \inlinecode{Leaf}, which represents the empty tree, or a \inlinecode{Node}, which is a tuple with three elements, 
the subtree to the left, the value for the node and the subtree to the right, respectively. The recursive depth-first search algorithm 
for trees can be implemented as:

\begin{gospell}
let rec depth_search t n = 
  match t with
  | Leaf -> false
  | Node (l,x,r) -> x = n || depth_search l n || depth_search r n
(*@
  r = depth_search t n
  variant t
  ensures List.mem n (to_list t) <-> r
*)
\end{gospell}

The following recursive function \inlinecode{depth\_search} shows how a tree is traversed through its depth firstly when trying to find
a node that has the same value as the value given in the arguments.

Since the algorithm searches the subtree to the left and subtree to the right it is known that each of those trees is smaller than their
parent. In the worst case, where the element is not found in the tree, we know that recursion terminates with the leaves, so the variant 
clause will be the tree that was searched. One way to guarantee the result is correct is to transform the tree into a list and checking if
the desired element is found in the resulting list with the \inlinecode{List.mem} operation.

To transform the tree into a list we formed an auxiliary logical function:

\begin{gospell}
(*@ function rec to_list (t: 'a tree) : 'a list = 
  match t with
  | Leaf -> []
  | Node l x r -> x :: to_list l @ to_list r
*)
(*@
  variant t
*)
\end{gospell}

The recursive function \inlinecode{to\_list} transforms a tree firstly from the own element of the root, following the left side and 
ending with the right side. Since, for this example, we are only concerned with the inclusion of an element in the list, the order 
in which it is constructed doesn't have effect on the outcome of the solution, our decision was arbitrary.

Similarly to the function \inlinecode{depth\_search} the termination can be proven by the tree itself.

The generated \cml code was:

\begin{cakeml}
'a datatype tree = Leaf | Node of 'a tree * 'a * 'a tree

fun depth_search t n = let val t1 = t in
  let val n1 = n in
  (case t1 of
    Leaf => false
  | Node l x r =>
    (x = n1) orelse ((depth_search l n1) orelse (depth_search r n1)))
\end{cakeml}

The definition of the tree data type is incorrect for three reason, number one the ordering of the syntax for the left side declaration of
a data type starts with the token \inlinecode{datatype} and only after the generic type \inlinecode{'a}. The second reason is the
unnecessary token \inlinecode{of} that is not used when declaring a tuple in \cml syntax. Thirdly the tokens \inlinecode{*} are also
unnecessary when declaring tuples. Finally, the let-bindings displayed the same errors as previous examples.

The corrected code below:

\begin{cakeml}
datatype 'a tree = Leaf | Node ('a tree) 'a ('a tree)

fun depth_search t n = let val t1 = t in
  let val n1 = n in
  (case t1 of
    Leaf => False
  | Node l x r =>
    (x = n1) orelse ((depth_search l n1) orelse (depth_search r n1)))
    end end
\end{cakeml}

Correcting the order and removing the unnecessary tokens in data type definition, achieves a more robust extraction mechanism that 
produces compilable \cml code.

\subsection{Tree Comparison}

With this example we only want to understand if the module system is working as intended in \cml. The module system in \cml is very 
limited since it does not support signatures and functors. We present a simple tree module in \ocaml that contains the comparison between
trees.

\begin{gospell}
module Tree = struct

  type 'a tree =
    | Leaf
    | Node of 'a tree * 'a * 'a tree

  let rec cmp t1 t2 =
    match t1, t2 with
    | Leaf, Leaf -> true
    | Leaf, _ -> false
    | _, Leaf -> false
    | Node (l1,x1,r1), Node (l2,x2,r2) -> cmp l1 l2 && x1 = x2 && cmp r1 r2
    
  (*@
  r = cmp t1 t2
  variant t1
  ensures r <-> t1 = t2
  *)
end
\end{gospell}

The function \inlinecode{cmp} uses pattern matching to compare the trees, in case both are leaves it is true, if one is a leaf, but 
the other is not they are different, finally if both are nodes then it depends on the respective root and subtrees.

Logically the result should be equal to comparing \inlinecode{t1} and \inlinecode{t2} and the variant clause could be either one of 
the trees since it only matters when the result is correct therefore the trees have the same structure.

Generated code to \cml was:

\begin{cakeml}
'a datatype tree = Leaf | Node of 'a tree * 'a * 'a tree

fun cmp t1 t2 = let val t11 = t1 in
  let val t21 = t2 in
  (case (t11, t21) of
    (Leaf, Leaf) => true
  | (Leaf, _) => false
  | (_, Leaf) => false
  | (Node l1 x1 r1, Node l2 x2 r2) =>
    (cmp l1 l2) andalso ((x1 = x2) andalso (cmp r1 r2)))
\end{cakeml}

Once more the data types and let-bindings expressions failed to extract correctly, but more notably the module is not present.
The corrected version is demonstrated below:

\begin{cakeml}
datatype 'a tree = Leaf | Node ('a tree) 'a ('a tree)

fun cmp t1 t2 = let val t11 = t1 in
  let val t21 = t2 in
  (case (t11, t21) of
    (Leaf, Leaf) => True
  | (Leaf, _) => False
  | (_, Leaf) => False
  | (Node l1 x1 r1, Node l2 x2 r2) =>
    (cmp l1 l2) andalso ((x1 = x2) andalso (cmp r1 r2)))
    end
  end
\end{cakeml}

For the same problems, the same solutions were implemented, unfortunately modules could not be tested for the \cml code, but now
it is evident they were not supported for this translation tool.

\section{Analyses For The Case Studies}

Without addressing exception scoping, array operation semantics, and let-binding syntax, the extracted code remains invalid 
and cannot serve as a verified executable target. Therefore, improving the extraction pipeline from \ocaml and \gospel to 
\cml necessitates careful handling of these language-specific constructs to ensure that the resulting code is not only 
syntactically correct but also preserves the behaviour of the original verified source.
